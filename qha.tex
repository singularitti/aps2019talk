% !TEX program = pdflatex
% !TEX encoding = UTF-8
% !BIB program = biber
% !TEX spellcheck = en_US
\documentclass[13pt,aspectratio=169]{beamer}

\usetheme{Berkeley}
\usecolortheme{whale}
\makeatletter
\beamer@headheight=1.5\baselineskip
\makeatother
\setbeamertemplate{navigation symbols}{}
\addtobeamertemplate{navigation symbols}{}{%
	\usebeamerfont{footline}%
	\usebeamercolor[fg]{footline}%
	\hspace{1em}%
	\insertframenumber/\inserttotalframenumber
}
\setbeamertemplate{caption}[numbered]
\setbeamercovered{transparent}

\usepackage[utf8]{inputenc}
\usepackage{amsmath}
\usepackage{amsfonts}
\usepackage{amssymb}
\usepackage{graphicx}
\usepackage[version=4]{mhchem}
\usepackage{siunitx}
\usepackage[labelformat=empty]{caption}
\usepackage{bm}
\usepackage[bibstyle=nature,backend=biber]{biblatex}

\newcommand*{\rmn}[1]{\romannumeral#1}
\newcommand*{\RMN}[1]{\uppercase\expandafter{\romannumeral#1}}
\makeatletter
    \def\@makefnmark{\hbox{{{\usebeamercolor[fg]{footnote mark}\usebeamerfont*{footnote mark} [\@thefnmark]}}}}

    \def\@makefntext#1{%
        \def\insertfootnotetext{ #1}%
        \def\insertfootnotemark{\@makefnmark}%
        \usebeamertemplate***{footnote}}
\makeatother
\DeclareCiteCommand{\footcite}[\mkbibfootnote]
	{\usebibmacro{prenote}}
	{\printnames[family-given]{labelname}%
		\newunit
		\printfield{journaltitle}%
		\newunit
		\printfield{year}
		\newunit
		\printlabeldateextra
	}
	{\addsemicolon\space}
	{\usebibmacro{postnote}
}
\DeclareCiteCommand{\footcitetext}[\footnotetext]
	{\usebibmacro{prenote}}
	{\printnames[family-given]{labelname}%
		\newunit
		\printfield{journaltitle}%
		\newunit
		\printfield{year}
	}
	{\addsemicolon\space}
	{\usebibmacro{postnote}
}

\author[Qi Zhang et. al.]{\underline{Qi Zhang} \inst{1} \and Tian Qin \inst{2} \and Renata Wentzcovitch\inst{1,3} \and Koichiro Umemoto\inst{4}}
\institute{\inst{1} Applied Physics and Applied Mathematics Department, Columbia University, New York, NY \and%
	\inst{2} Department of Earth Sciences, University of Minnesota, Minneapolis, MN \and%
	\inst{3} Lamont-Doherty Earth Observatory, Columbia University, Palisades, NY \and%
	\inst{4} Earth-Life Science Institute, Tokyo Institute of Technology}
\title[\texttt{qha}]{\texttt{qha}: A Python package for quasi-harmonic free energy calculation for multi-configuration systems\footcite{qin2018qha}}
\date{}

\addbibresource{ref.bib}

\begin{document}

\begin{frame}
	\titlepage
\end{frame}

\section{Introduction}

\subsection{Quasi-harmonic approximation (QHA)}
\begin{frame}{\subsecname}
	\begin{itemize}[<+(1)->]
		\setlength\itemsep{1em}
		\item a useful tool to compute materials thermodynamic properties at high $T$, $P$
		\item a good approximation when $T$ is not too close to $T_\text{M}$
		\item Born--Oppenheimer approximation: static contribution + vibrational contribution (QHA on phonon spectra)
		\item deal with multiple configuration systems
	\end{itemize}
\end{frame}

\subsection{Multi-configuration systems that \texttt{qha} may apply}
\begin{frame}{\subsecname}
	\begin{itemize}[<+(1)->]
		\setlength\itemsep{1em}
		\item the order-disorder phase boundary between ice-\RMN{8} and ice-\RMN{7} \footcite{umemoto2010order}
		\item the relative stability of hydrous defects in \ce{Mg2SiO4}-forsterite at high $P$ and $T$ \footcite{qin2018ab}
		\item the effect of disorder and iron concentration on the spin crossover diagram of \ce{Fe^3+}-bearing \ce{MgSiO3}-bridgmanite \footcite{shukla2016spin}
	\end{itemize}
\end{frame}

\subsection{Methods}
\begin{frame}{\subsecname}
	\begin{itemize}[<+(1)->]
		\item single configuration system:
		      \begin{align}
			      Z(T, v) & = \exp\big( -{E(v)}/{k_B T} \big) \prod_{\bm{q}, s} \frac{\exp(-\hbar \omega_{\bm{q}s}(v)/2k_B T)}{1 - \exp(-\hbar \omega_{\bm{q}s}(v)/k_B T)}, \\
			      F(T, v) & = -k_B T \ln Z(T, v).
		      \end{align}
		\item multi-configuration systems:
		      \begin{align}
			      Z(T, v) & = \sum_{n=1}^{N_c} g_n Z_n (T, v), \\
			      F(T, v) & = -k_B T \ln Z(T, v).
		      \end{align}
		\item finite strain equation of state fitting to string $\{F(T, v)\}$’s to a continuous function $F(T, V)$
	\end{itemize}
\end{frame}

\begin{frame}{Flow chart}
	\centering
	\includegraphics[height=0.95\textheight]{images/flow}%
\end{frame}

\section{Examples}

\subsection{Ice VII(disordered)-VIII(ordered) phase transition}
\begin{frame}{\subsecname{} \RMN{1}}
	\centering
	\includegraphics[height=0.9\textheight]{images/ice7}%
\end{frame}

\begin{frame}{\subsecname{} \RMN{2}}
	\begin{figure}
		\includegraphics[height=0.75\textheight]{images/ice_prob}%
		\caption{Possibilities $P_i(V, T) = \frac{g_i \exp\Big(\frac{-E_i(V)}{k_B T}\Big)}{Z(V, T)}$ of the $52$ configurations}
	\end{figure}
\end{frame}

\begin{frame}{\subsecname{} \RMN{3}}
	\begin{columns}
		\begin{column}{0.4\textwidth}
			\begin{figure}[t]
				\includegraphics[width=\columnwidth]{images/cp}%
			\end{figure}
		\end{column}
		\hfill
		\setcounter{footnote}{1}
		\begin{column}{0.5\textwidth}
			\begin{figure}[t]
				\includegraphics[width=\columnwidth]{images/boundary}%
				\caption{\footnotemark Blue line: static, red lines: QHA results, black lines: experimental results,
					solid lines: \ce{H2O}, dashed lines: \ce{D2O}}
			\end{figure}
		\end{column}
	\end{columns}
	\footcitetext{umemoto2010order}
	\setcounter{footnote}{4}
\end{frame}

\section{Conclusion}

\subsection{Comparisons with other codes}
\begin{frame}{\subsecname}
	\begin{columns}
		\begin{column}{0.45\textwidth}
			\texttt{qha}\\
			\begin{itemize}[<+(1)->]
				\item addresses the thermodynamic properties of multi-configuration systems
				\item directly samples the free energy in Brillouin zone
				\item balances the speed and flexiblility
			\end{itemize}
		\end{column}

		\begin{column}{0.45\textwidth}
			other codes\\
			\begin{itemize}
				\item address single configuration systems
				\item integrate the vibrational density of states $g(\omega)$ to get $F$\footnotemark
				\item less flexible and extensible\footnotemark
			\end{itemize}
		\end{column}
	\end{columns}
	\setcounter{footnote}{5}
	\footcitetext{Petretto:2018gg}
	\stepcounter{footnote}
	\footnotetext{\url{https://github.com/dalcorso/thermo_pw}}
\end{frame}

\subsection{Applications or extensibilities}
\begin{frame}{\subsecname}
	\begin{columns}
		\begin{column}{0.45\textwidth}
			\begin{figure}
				\includegraphics[height=0.7\textheight]{images/website}%
				\captionsetup{labelformat=empty}
				\caption{\scriptsize\url{https://github.com/MineralsCloud/qha}}
			\end{figure}
		\end{column}

		\begin{column}{0.45\textwidth}
			\begin{itemize}[<+(1)->]
				\item Extensibilities to materials research
				      \begin{itemize}
					      \item thermoelastic properties of materials\footnotemark
					      \item metals with phonon frequencies varying at different electronic temperatures (introduced in seission H17.00005, Tuesday)
					      \item calculate the geotherm and isentrope\footnotemark
				      \end{itemize}
			\end{itemize}
		\end{column}
	\end{columns}
	\setcounter{footnote}{7}
	\footcitetext{Wu:2011ea}
	\stepcounter{footnote}
	\footcitetext{Cardona:2017dd}
\end{frame}

\begin{frame}{\secname}
	\begin{itemize}[<+(1)->]
		\setlength\itemsep{1em}
		\item \texttt{qha} is a fast, user-friendly Python package for tranditional QHA calculations
		\item \texttt{qha} can calculate multi-configuration system's equation of state and other thermodynamic properties
		\item \texttt{qha} is extensible and will form a more complete toolchain in the near future
	\end{itemize}
\end{frame}

\section{Acknowledge\-ments}
\begin{frame}{\secname}
	\begin{itemize}
		\item NSF EAR-1503084, NSF EAR-1341862, NSF EAR-1348066
		\item Stampede2 at the Texas Advanced Computing Center (TACC), University of Texas at Austin
		\item Habanero HPC Cluster, Columbia University
	\end{itemize}
\end{frame}

\section{References}
\begin{frame}[allowframebreaks]{\secname}
	\printbibliography
\end{frame}

\end{document}