% !TEX program = pdflatex
% !TEX encoding = UTF-8
% !BIB program = bibtex
% !TEX spellcheck = en_US
\documentclass[13pt,aspectratio=169]{beamer}

\usetheme{Berkeley}
\usecolortheme{crane}
\makeatletter
\beamer@headheight=1.5\baselineskip
\makeatother
\setbeamertemplate{navigation symbols}{}
\addtobeamertemplate{navigation symbols}{}{%
	\usebeamerfont{footline}%
	\usebeamercolor[fg]{footline}%
	\hspace{1em}%
	\insertframenumber/\inserttotalframenumber
}
\setbeamertemplate{caption}[numbered]
\setbeamercovered{transparent}

\usepackage[utf8]{inputenc}
\usepackage{amsmath}
\usepackage{amsfonts}
\usepackage{amssymb}
\usepackage{graphicx}
\usepackage[version=4]{mhchem}
\usepackage{siunitx}
\usepackage[figurename=Fig.]{caption}
\usepackage{bm}
%\usepackage[citestyle=verbose,bibstyle=numeric,backend=biber]{biblatex}

\newcommand*{\rmn}[1]{\romannumeral#1}
\newcommand*{\RMN}[1]{\uppercase\expandafter{\romannumeral#1}}

\author[Qi Zhang et. al.]{\underline{Qi Zhang} \inst{1} \and Tian Qin \inst{2} \and Renata Wentzcovitch\inst{1,3} \and Koichiro Umemoto\inst{4}}
\institute{\inst{1} Applied Physics and Applied Mathematics Department, Columbia University, New York, NY \and%
	\inst{2} Department of Earth Sciences, University of Minnesota, Minneapolis, MN \and%
\inst{3} Lamont-Doherty Earth Observatory, Columbia University, Palisades, NY \and%
\inst{4} Earth-Life Science Institute, Tokyo Institute of Technology}
\title[\texttt{qha}]{\texttt{qha}: A Python package for quasi-harmonic free energy calculation for multi-configuration systems\cite{qin2018qha}}
\date{}

%\addbibresource{ref.bib}

\begin{document}

\begin{frame}
	\titlepage
\end{frame}

\section{Introduction}

\subsection{Quasi-harmonic approximation (QHA)}
\begin{frame}{\subsecname}
	\begin{itemize}[<+(1)->]
		\setlength\itemsep{1em}
		\item a useful tool to compute materials thermodynamic properties at high $T$, $P$
		\item a good approximation up to about $\frac{ 2 }{ 3 }$ of the $T_\text{melting}$
		\item static contribution by (DFT, AIMD), vibrational contribution: DFPT followed by QHA
		\item deal with multiple configuration systems
	\end{itemize}
\end{frame}

\subsection{Multi-configuration system that \texttt{qha} may apply}
\begin{frame}{\subsecname}
	\begin{itemize}[<+(1)->]
		\setlength\itemsep{1em}
		\item the order-disorder phase boundary between ice-\RMN{8} and ice-\RMN{7} \cite{umemoto2010order},
		\item the relative stability of hydrous defects in \ce{Mg2SiO4}-forsterite at high $P$ and $T$ \cite{qin2018ab},
		\item the effect of disorder and iron concentration on the spin crossover diagram of \ce{Fe^3+}-bearing \ce{MgSiO3}-bridgmanite \cite{shukla2016spin}.
	\end{itemize}
\end{frame}

\subsection{Methods}
\begin{frame}{\subsecname}
	\begin{itemize}[<+(1)->]
		\item single configuration system:
		      \begin{align}
			      Z(T, V) & = \exp\big( -{E(V)}/{k_B T} \big) \prod_{\bm{q}, m} \frac{\exp(-\hbar \omega_{\bm{q}s}/2k_B T)}{1 - \exp(-\hbar \omega_{\bm{q}s}/k_B T)}, \\
			      F(T, V) & = -k_B T \ln Z(T, V),
		      \end{align}
		\item multi-configuration systems:
		      \begin{align}
			      Z(T, V_i) & = \sum_{n=1}^{N_c} g_n Z_n (T, V_i), \\
			      F(T, V_i) & = -k_B T \ln Z(T, V_i).
		      \end{align}
		\item finite strain equation of state fitting to string $\{F(T, V_i)\}$’s to a continuous function $F(T, V)$
	\end{itemize}
\end{frame}

\begin{frame}{\subsecname}
	\centering
	\includegraphics[height=0.95\textheight]{images/flow}%
\end{frame}

\section{Examples}

\subsection{ice VII(disorder)-VIII(order) phase transition}
\begin{frame}{\subsecname}
	\centering
	\includegraphics[height=0.9\textheight]{images/ice7}%
\end{frame}

\begin{frame}{\subsecname}
	\begin{figure}
		\includegraphics[height=0.8\textheight]{images/ice_prob}%
		\caption{Possibilities $P_i(V, T) = \frac{g_i \exp\Big(\frac{-E_i(V)}{k_B T}\Big)}{Z(V, T)}$ of the $52$ configurations}
	\end{figure}
\end{frame}

\begin{frame}{\subsecname}
	\begin{columns}
		\begin{column}{0.45\textwidth}
			\begin{figure}
				\includegraphics[width=\columnwidth]{images/deltae}%
				\caption{Histograms of the total energy $E_i(V)$ at several volumes at
					(a) $\sim$\SI{5}{\giga\pascal},
					(b) \SI{50}{\giga\pascal}.}
			\end{figure}
		\end{column}

		\begin{column}{0.45\textwidth}
			\begin{figure}
				\includegraphics[width=\columnwidth]{images/dltevsp}%
				\caption{Energy distribution: $\Delta E(V) = \max_{1 \leq i \leq 52}[E_i] - \min_{1 \leq i \leq 52}[E_i]$\\
					Clapeyron slope: $\Big( \frac{ \partial T }{ \partial P } \Big)_{ S } = \Big( \frac{ \partial V }{ \partial S } \Big)_{ P } < 0$}
			\end{figure}
		\end{column}
	\end{columns}
\end{frame}

\subsection{stability of hydrous defects in \ce{Mg2SiO4}-forsterite}
\begin{frame}[allowframebreaks]{\subsecname}
	\begin{columns}
		\begin{column}{0.4\textwidth}
			\begin{figure}
				\includegraphics[width=\columnwidth]{images/si}%
				\caption{Configurations of \ce{(4H)^X_{Si}}
				defects. Pink polyhedra represent vacant \ce{Si} sites.}
			\end{figure}
		\end{column}

		\begin{column}{0.4\textwidth}
			\begin{figure}
				\includegraphics[width=\columnwidth]{images/mg}%
				\caption{Configurations of \ce{(2H)^X_{Mg}}
				defects. Green polyhedra represent vacant \ce{Mg} sites.}
			\end{figure}
		\end{column}
	\end{columns}

	\begin{figure}
		\includegraphics[height=0.6\textheight]{images/simg}%
		\caption{Degeneracies, relative energies, and probabilities of various defects (static calculation)}
	\end{figure}
\end{frame}

\section{Conclusion}

\subsection{Comparisons with other codes}
\begin{frame}{\subsecname}
	\begin{columns}
		\begin{column}{0.45\textwidth}
			\texttt{qha}\\
			\begin{itemize}[<+(1)->]
				\item addresses multi-configuration systems
				\item $G(T,p) = F(T, V) - \Big( \frac{ \partial F }{ \partial V } \Big)_T V$
				\item directly sample the free energy in Brillouin zone
				\item has command line interface for running and drawing
				\item uses JIT techniques to speedup computation
			\end{itemize}
		\end{column}

		\begin{column}{0.45\textwidth}
			other codes\\
			\begin{itemize}
				\item address the thermodynamic properties of single configuration systems
				\item $G(T,p)= \min_{V}[F(T,V)+pV]$ \cite{phonopy}
				\item integrate the vibrational density of states $g(\omega)$ to get $F$ \cite{Petretto:2018gg}
			\end{itemize}
		\end{column}
	\end{columns}
\end{frame}

\subsection{Applications to geophysical and mineralogy research}
\begin{frame}{\subsecname}
	\begin{columns}
		\begin{column}{0.45\textwidth}
			\begin{figure}
				\includegraphics[height=0.8\textheight]{images/website}%
				\captionsetup{labelformat=empty}
				\caption{\scriptsize\url{https://github.com/MineralsCloud/qha}}
			\end{figure}
		\end{column}

		\begin{column}{0.45\textwidth}
			\begin{itemize}[<+(1)->]
				\item investigate thermoelastic properties of materials \cite{Wu:2011ea}
				\item investigate metals with its phonon frequencies varying at different temperature
				\item calculate the geotherm and isentrope \cite{Cardona:2017dd}
				\item calculate the isotope ratio and isotope fractionation factor
			\end{itemize}
		\end{column}
	\end{columns}
\end{frame}

%\section{References}
%\begin{frame}[allowframebreaks]{\secname}
%	\printbibliography
%\end{frame}
\bibliography{ref}

\begin{frame}{Acknowledgements}
	\begin{itemize}
		\item NSF EAR-1503084, NSF EAR-1341862, NSF EAR-1348066
		\item Stampede2 at the Texas Advanced Computing Center (TACC), University of Texas at Austin
	\end{itemize}
\end{frame}

\end{document}